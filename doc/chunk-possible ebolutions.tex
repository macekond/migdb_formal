There are several way how the evolution of ORM system can be implemented. The way when the application is evolved, database scheme re-genareted by an ORM framework and data migrated manually is the most common nowadays. The goal of our project is to minimize the manual work in the process of data evolution. We have following options:
\begin{itemize}
	\item Backward round-trip engineering can be used. It means the application and database is evolved and then the difference in structure is used to define data migration. This approach is inappropriate for situations where there is stored data as a change in database schema can be connected with more possible changes of data. Hence an additional information from user will be needed to migrate data correctly.
    \item Forward engineering can be used in two ways. First approach evolves application, re-generate a database schema and generate a data migration according to definition of application evolution. Second approach uses the definition of application evolution as a source for evolution of database schema and data migration. In contrast this approach preserves the original database - it only alters its structure, whereas the first approach creates a new database.
\end{itemize}
All approaches can be used and all of them have their pros and cons. We decide to use forward engineering approach as it is a natural way of evolution. Next we decide to generate both migrations (for database schema and data) from definition of application evolution. This allows us to keep history of all layers and to verify the evolution on multiple levels. 
