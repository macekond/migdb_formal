\documentclass[11pt]{article}

\begin{document}
\section{Related Work}



\subsection{Database Evolution}
The problem of database schema evolution is well known issue which is discussed for a long time, nevertheless the term "database refactoring" is quite new. This section provides an overview of significant related work in area of database evolution and refactoring. 

\subsubsection{Evolution of Relational Database}
\paragraph{Informal definitions} The taxonomy of relational database evolution based on the entity-relationship model  is proposed in \cite{Roddick:TaxonomyOnERM}. The evolution is described as a change in entity-relationship model and change in relational database.  The semantics change patterns in context of conceptual schema is described in \cite{Wedemeijer:SemanticChangePatternsInConceptualSchema}, although its impact on database schema or data is not described. The main cases of data evolution are defined in both publications, however the description is informal. The extensive set of possible database refactorings is provided in \cite{Ambler:DbRefactoringBook}, where both schema and data evolution is discussed. The refactorings are intended to be used by database administrators, thus it assume database-first approach to evolution whereas this proposal is application-first.   

\paragraph{Formal framework} A general formal framework for database evolution is defined in \cite{McBrien:formal-framework-transformation}. The framework is based on a set of basic graph transformations which are then extended to transformations of the entity-relationship model. The framework and the defined transformations can be used for comparison with our framework, where our framework consider stored data. The contribution of the formal framework is definition of equivalent structures in schemas.

\paragraph{Tools for Database Evolution} The MeDEA project \cite{Meda:Main}, \cite{Meda:MDDViewBasedEv} offers a tool for evolution of both database schema and stored data based on model-driven approach. Both MeDEA and proposed framework allows automatic database evolution, the difference is in different target audience.  The framework is aimed to be used by database administrators, therefore it is aimed on database structures (such as views) which are not considered in our approach as it is aimed to be used by developers. Proposed framework can use MeDEA to enrich its ability on database model, whereas MeDEA can use the code-centric point of view to design developer friendly transformations.

The PRISM is a research project for data management under schema evolution \cite{PRISM} in contrast with our proposal and with MeDEA it uses it extends the SQL command set by so called schema modification operators which implements the schema evolution. The project is aimed to  be used by database administrators (as well as MeDEA). The PRISM's set of modification operators could be used for database transformation in our proposal, however we prefer to work with a standard SQL set of commands. 

\paragraph{ORM Frameworks} here are many object-relational mapping frameworks available for developers, and some of them provide tools for database migration. Hibernate \cite{Hibernate} is one of the most popular ORM frameworks in the Java community.  It provides customizable ORM for a wide range of databases, however it does not support complex database migrations.  It is capable only of creating a new table or adding a new column, hence it is not possible to, for example, drop a table or copy values from one column to another. Active Record \cite{Active_Record} is an ORM framework in the Ruby on Rails environment. Since its first version it has contained support for database evolution according to the create-update-delete principle, in the form of so-called migrations \cite{Rails:Migrations} which can be extended by adding user SQL commands. Entity Framework \cite{Entity_Framework} is Microsoft's ORM solution for the .NET platform. Its capabilities of data evolution support are similar to those of Active Record. 

Each of presented ORM frameworks provides support for creating data evolution, as they are all capable of adding a new class as an entity and creating a corresponding table in the database. On the other hand, only two of them (Active Record and Entity Framework) are capable of updating and deleting evolution. In the case of evolutions which need to manipulate data, e.g. moving the property, none of the frameworks mentioned provides a built-in solution. Active Record and Entity Framework allow developers to describe the data move manually. In contrast to these ORMFs, our proposal  provides a higher level of user comfort because of its capability to handle complex data evolutions.

\subsubsection{Evolution of Object-Oriented Databases}
The issue of database evolution is discussed not only for relational databases, but for object-oriented databases too. There is proposed an approach for  schemes of object-oriented databases in \cite{Banerjee:SemanticsOfSchemaEvolutionInOODB} which based on a set of schema invariants (representing the properties of the schema) and a set of rules for invariant preserving schema changes. The impact of evolution on instances is considered too. The approach is presented informally, while our approach is based on set of formal definitions.

The formal definition of evolution of object-oriented databases is presented in \cite{Peters:AxiomaticModelOfDynamicEvolutionInOODB}. The approach is based on type system representing the database schema, on axioms which defines the schema structural constraints and on changes which evolves the system. There are three change operators: add, remove, modify defined in the paper. The model satisfies the properties of soundness and completeness. The approach in this paper is based on small transformations (changes) too, but its aim is to combine them in a way advanced refactoring cases can be easily applied on a software. 

The SERF framework \cite{Claypool:SERF} provides a similar functionality to the user of an object-oriented database as we do for the user of a software based on ORM. The approach is based on templates a user can create and maintain. Limitation and condition of some evolution cases is discussed in \cite{Claypool:SERF-formal}.

All mentioned approaches address similar  issue as this paper, in contrast it focusses on the homogenous environment of object-oriented databases, whereas our approach aims the heterogenous world of ORM.  



The formal approach based on epistemic logic (logic of knowledge) is introduced in \cite{Chang-LogicFrameworkForDbRefactoring}. The approach presents a database as a set of beliefs (knowledge), which is revision can be seen as refactoring. The approach consider not only database structure but database queries too, on the other hand data are not considered in this approach.

\subsection{Data Evolution Metrics}
There are proposed several metrics for classification of impact of a change on the conceptual schema in \cite{Wedemeijer:ConceptualMetrics}. These metrics can help us to describe the impact of proposed transformation on a software which is represent as two models (application and database). Only some metrics could be used for describing transformations as the others describe the state of conceptual schema. The universe (domain) is the application as all changes origins on this level.

The first metric "\textbf{Justified change}" means (in our context) that for each change of database exists a change of application. In case the evolution of the software was based on transformations proposed in this paper the metric is equal to one - each change is justified.

The metric "\textbf{Proportional change}" which compares the size and severity of change between application and database level can be used to describe complexity of transformations. The terms size of change and complexity of change has to be defined in order to use the metric. The size of change is equal to the number of entities in the model and instances which are affected by a change. The complexity of change is measured by number of atomic operations on each level needed to fulfill the evolution. This metrics is used to classify every single transformation.

The metric "\textbf{Impact on Complexity}" determines how the complexity of software or its components  changed. The metric can be used to evaluate each single transformations

\subsection{Impact of Database Evolution on Application}
\paragraph{Prediction of database change impact on application}The analysis technique for identifying the impact of relational database change upon object oriented application is proposed in \cite{Maule:ImpactAnalysis}. The technique focus on relational database and its change which impact on application is predicted. The analysis technique is able to identify changes in application structure as well as changes in database queries used in application. In contrast approach proposed in this paper does not consider database queries and theirs change and it is not mentioned to predict the impact of a change, but to  
process the change in the whole software, although it can be used for simulation of a change on formal level.

\bibliographystyle{plain}
\bibliography{references}
\end{document}
